% ********** Rozdział 1 **********
\chapter{Wymagania projektu}
\section{Wymagania funkcjonalne i niefunkcjonalne}
\subsection{Wymagania funkcjonalne}

Z założeń projektu można powiedzieć, że funkcjonalne wymagania są:
\begin{itemize}
   \item Założenie konta
   \item Łatwe zarządzanie obszarzem bazy danych 
\item Sprawdzanie aptek w pobliżu
\item Sprawdzanie informacje o lekach
\item Wyświetlanie informacji o lekarzach
\item Recepty na leki 
\item Zważanie na przeciwwskazania
\item Głównie - wygodna, intuicyjnie zrozumiała nawigacja w aplikacji
\end{itemize}

W dodatek do tej listy można dodać bardziej szczegółowy opis. Założenie konta będzie potrzebować wprowadzanie i zapisywanie nowych danych do bazy danych.\newline Korzystanie z bazy danych do wyświetlenia nazwy, adresu i dt. informacje o wszystkich aptekach miasta; \newline Wyświetlanie danych o lekach, jako nazwa, producent, cena, dostępność, dawka, przeciwwskazania z uwagi na alergie. \newline Lekarze - imie, nazwisko, specjalność.
\subsection{Wymagania niefunkcjonalne}\label{przypisy}. 

Najbardziej wymagania niefunkcjonalne będą polegać na umówieniach z pracownikami i kierownictwem aptek, szpitalów i przychodzień. Jedną z najważniejszych możliwości, które musi udostępnić projekt "E-Szpital", jest łatwa komunikacja informacyjna w bazie danych, powiązana z każdym budynkiem, prowadzającym pomoc medyczną. \newline



% ********** Koniec rozdziału **********
